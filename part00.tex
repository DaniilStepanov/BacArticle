%%%%%%%%%%%%%%%%%%%%%%%%%%%%%%%%%%%%%%%%%%%%%%%%%%%%%%%%%%%%%%%%%%%%%%%%%%%%%%%
\section{Introduction}
%%%%%%%%%%%%%%%%%%%%%%%%%%%%%%%%%%%%%%%%%%%%%%%%%%%%%%%%%%%%%%%%%%%%%%%%%%%%%%%
Software quality is very important aspect of software development because in our days software use in many areas of human life. Errors in modern software can lead to fatal consequences. Methods of software quality improving consists of programmers bugs fixing. Bounded model checking~\cite{biere2003bounded} is one of the most popular methods of software quality improving. In this method program represented as first order logic formula. Next this formula is sends in the SMT-solver, which makes conclusion about it's satisfiability. Thus considered full space of program states, aimed to find states, which brakes security properties, for example, outing bounds of array. One of the hardest questions in BMC is interprocedural analysis~--- how function calls mapped in the program state~\cite{InterprocIsHard}. As default in BMC perfomed function inlining, which often makes analyze impossible because of size of the resulting program state.

This work represents a method of resolving interprocedural analysis problem in BMC through function approximation --- short description of function behavior. A function approximation expressed in terms of its arguments and return value. Main advantage of using approximation is that they decrease the size of resulting formulae. Prototype is based on bounded model checker Borealis~\cite{Borealis}, but the method of approximation can be applied for another BMC tools. Evaluation results show that our approach is able to give some positive effect to BMC process.

The rest of the paper is organized as follows. We lay the foundation for our work by introducing Borealis and bounded model checking in section~\ref{sec:basics}. The process of taking functions approximations explained in section~\ref{sec:mining}. We talk about preliminary evaluation results and related work in sections~\ref{sec:evaluation} and~\ref{sec:related-work} respectively.
